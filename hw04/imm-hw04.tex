\documentclass[10pt]{jhwhw}
\author{Ian Malerich}
\title{Math S 373: Homework 4}
\usepackage{amssymb, amsfonts, amsmath, mathtools, graphicx, breqn, soul}
\usepackage{minted, subfig, float, scrextend, setspace, amsthm}
\usemintedstyle{friendly}

\begin{document}
\raggedright

%% Problem 1
\problem{} (30 points)

	\begin{enumerate}
		\item Use the forward-difference and backward-difference formulas to determine each missing entry
			in the following table with $f(x) = \sin x$, compute the errors and find error
			bounds using error formulas. \\
		\item Choose your favorite function $f$, nonzero number $x$.
			Generate approximations of $f'_n(x)$ to $f'(x)$ by
			$$
				f'_n(x) = \frac{f(x + 10^{-n}) - f(x)}{10^{-n}}
			$$
			for $n = 1,2,\ldots,10$ and describe what happens.
	\end{enumerate}

\solution

\part
	$$
	\begin{tabular}{|c|c|c|c|c|}
		\hline
		$x$ & $f(x)$ & $f'(x)$ & \cos x  & \varepsilon = \mid\cos x - f'(x)\mid \\ \hline
		0.5 & 0.4794 & 0.85200 & 0.87758 & 0.025580 \\ \hline
		0.6 & 0.5646 & 0.79600 & 0.82534 & 0.029340 \\ \hline
		0.7 & 0.6442 & 0.79600 & 0.76484 & 0.031160 \\ \hline
	\end{tabular}
	$$

	\bigbreak
	We find the error term $\mid \frac{h}{2}f''(\xi)\mid$ from the following formula
	$$
		f'(x_0) = \frac{f(x_0+h) - f(x_0)}{h} - \frac{h}{2}f''(\xi)
	$$
	This error bound will be the same for forward and backward difference.
	We have $f''(x) = -\sin x$  and $h = 0.1$ in all cases, thus our error is given by
	$\mid \frac{0.1}{2}\sin \xi \mid$ where $\xi$ is a value within the bounds of our difference.
	For each, we therefore want to maximize the value $\sin \xi$.
	
	\bigbreak
	\textbf{$f'(0.5)$:} $max_{0.5,0.6}\ \sin \xi = 0.564642 \Rightarrow \upepsilon = 
	\frac{0.1}{2}0.564642 = 0.028232$ \\
	Thus we have an error bound of $0.02832$ which is just a bit over our actual error 
	of $0.025580$.

	\bigbreak
	\textbf{$f'(0.6)$:} $max_{0.6,0.7}\ \sin \xi = 0.644218 \Rightarrow \upepsilon = 
	\frac{0.1}{2}0.644218 = 0.03221$ \\
	Thus we have an error bound of $0.03221$ a little bit worse than before. Our actual
	error of $0.029340$ fits nicely within this bound.

	\bigbreak
	\textbf{$f'(0.7)$:} $max_{0.6,0.7}\ \sin \xi = 0.644218 \Rightarrow \upepsilon = 
	\frac{0.1}{2}0.644218 = 0.03221$ \\
	Since we are taking the backward difference, we come to the same error bound
	as the forward difference for $f'(0.6)$, that is $0.03221$. Our actual error
	for this value is the worst at $0.031160$ but still within the error bounds
	like we would expect.

\part

	Let $f(x) = \sin(x)$, and $x = 3$. \\
	We have $f'(x) = \cos(x)$, thus $f'(3) = \cos(3) = -0.9899924966$. \\

	$$
	\begin{tabular}{|c|c|c|c|}
		\hline
		n & $f'_n(x)$ & $\mid f'_n(x) - f'(x) \mid$ \\ \hline
		1 & -0.995393456265767 & 0.005400959665322 \\ \hline
		2 & -0.990681590968298 & 0.000689094367853 \\ \hline
		3 & -0.990062891599724 & 0.000070394999279 \\ \hline
		4 & -0.989999550952969 & 0.000007054352524 \\ \hline
		5 & -0.989993202188399 & 0.000000705587954 \\ \hline
		6 & -0.989992567285158 & 0.000000070684713 \\ \hline
		7 & -0.989992501865267 & 0.000000005264821 \\ \hline
		8 & -0.989992490763036 & 0.000000005837409 \\ \hline
		9 & -0.989992560151975 & 0.000000063551530 \\ \hline
		10 & -0.989992532396400 & 0.000000035795954 \\ \hline
	\end{tabular}
	$$

	In general, for larger values of $n$ we would expect the error between
	our approximation and the actual derivative to go down. However, if we 
	in the above example we can see the error actually increase from $n=7$ to
	$n=8$ and from $n=8$ to $n=9$. We do see a decrease again from $n=9$ to $n=10$
	however we still don't have the precision found at $n=7$ right before our estimates
	started to get worse.

%% Problem 2
\problem{} (60 points)

	\begin{enumerate}
		\item Approximate the following integral using the Trapezoidal rule, find a bound for the
			error using error formula and compare this to the actual error:
			$$
				\int_{0.5}^{1} x^4 dx.
			$$
		\item Repeat part (a) using Simpson's rule.
		\item Repeat part (a) using Composite Trapezoidal rule with $n=4$.
		\item Repeat part (a) using Composite Simpson rule with $n=4$.
		\item Write a code to implement part (c) and (d) in MATLAB.
		\item Write a function $v = $ CompositeTrapezoidalRule($f,a,b,n$) to implement
			Composite trapezoidal rule with a given $n$ for
			$$
				\int_a^b f(x) dx,
			$$
			verify your code with part (c).
	\end{enumerate}

\solution

%% Problem 3
\problem{} (20 points)

	\begin{enumerate}
		\item Show that the following quadrature formula has a degree of precision equal to 3,
			$$
				\int_{-1}^{1} f(x) dx = f(-\frac{\sqrt{3}}{3}) + f(\frac{\sqrt{3}}{3}).
			$$
		\item Approximate the following integral using Gaussian quadrature with $n= 2,3,5$,
			$$
				\int_1^{1.5} x^2 \ln x dx
			$$
	\end{enumerate}

\solution

%% Problem 4
\problem{} (20 points)

	\begin{enumerate}
		\item Use Composite Simpson rule with $n=4$ to approximate the following integral
			$$
				\int_0^1 x^{-1/4} \sin xdx
			$$
		\item Use Composite Simpson rule with $n=6$ to approximate the following integral
			$$
				\int_1^{\infty} \frac{\cos x}{x^3} dx
			$$
	\end{enumerate}

\solution

%% Problem 5
\problem{} (30 points)

	\begin{enumerate}
		\item Use the Gaussian Elimination with Backward substitution to solve the following
			linear system (must show intermediate steps).
			\begin{alignat*}{2}
				2x_1 &- 1.5x_2 &+ 3x_3 &= 1 \\
				-x_1 &         &+ 2x_3 &= 3 \\
				4x_1 &- 4.5x_2 &+ 5x_3 &= 1 \\
			\end{alignat*}
		\item Repeat (a) using the Gaussian Elimination with partial pivoting and Backward
			substitution (must show intermediate steps).
		\item Repeat (a) using Gaussian Elimination with scaled partial pivoting and 
			Backward substitution (must show intermediate steps).
	\end{enumerate}

\solution

%% Problem 6
\problem{} (20 points)

	\begin{enumerate}
		\item Solve the following linear system with forward and backward substitution,
			$$
			\begin{bmatrix}
				1 & 0 & 0 \\
				2 & 1 & 0 \\
				-1 & 0 & 1 \\
			\end{bmatrix}
			\begin{bmatrix}
				2 & 3 & -1 \\
				0 & -2 & 1 \\
				0 & 0 & 3 \\
			\end{bmatrix}
			\begin{bmatrix}
				x_1 \\ x_2 \\ x_3 \\
			\end{bmatrix} = 
			\begin{bmatrix}
				2 \\ -1 \\ 1
			\end{bmatrix}
			$$
		\item Use the LU factorization (no permutation) with forward and backward
			substitution to solve the following linear system (Use the sample MATLAB
			codes posted on classpage as discussed in class),
			\begin{alignat*}{2}
				2x_1 &+ 3x_2 &- x_3 &= 2 \\
				4x_1 &+ 4x_2 &- x_3 &= -1 \\
				-2x_1 &- 3x_2 &+ 4x_3 &= 1
			\end{alignat*}
	\end{enumerate}

\solution

%% Problem 7
\problem{} (20 points)

	\begin{enumerate}
		\item Use the $A = LDL^t$ factorization to solve the following linear system,
			\begin{alignat*}{2}
				2x_1 &- x_2 & &= 2 \\
				-x_1 &+ 2x_2 &- x_3 &= -1 \\
				&- x_2 &+ 2x_3 &= 1
			\end{alignat*}
		\item Repeat (a) with Cholesky factorization $A = LL^t$.
	\end{enumerate}

\solution

\end{document}
