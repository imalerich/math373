\documentclass[10pt]{jhwhw}
\author{Ian Malerich}
\title{Math S 373: Homework 4}
\usepackage{amssymb, amsfonts, amsmath, mathtools, graphicx, breqn, soul}
\usepackage{minted, subfig, float, scrextend, setspace, amsthm}
\usemintedstyle{friendly}

\begin{document}
\raggedright

%% Problem 1
\problem{} (30 points)

	\begin{enumerate}
		\item Use the forward-difference and backward-difference formulas to determine each missing entry
			in the following table with $f(x) = \sin x$, compute the errors and find error
			bounds using error formulas. \\
		\item Choose your favorite function $f$, nonzero number $x$.
			Generate approximations of $f'_n(x)$ to $f'(x)$ by
			$$
				f'_n(x) = \frac{f(x + 10^{-n}) - f(x)}{10^{-n}}
			$$
			for $n = 1,2,\ldots,10$ and describe what happens.
	\end{enumerate}

\solution

\part
	$$
	\begin{tabular}{|c|c|c|c|c|}
		\hline
		$x$ & $f(x)$ & $f'(x)$ & \cos x  & \varepsilon = \mid\cos x - f'(x)\mid \\ \hline
		0.5 & 0.4794 & 0.85200 & 0.87758 & 0.025580 \\ \hline
		0.6 & 0.5646 & 0.79600 & 0.82534 & 0.029340 \\ \hline
		0.7 & 0.6442 & 0.79600 & 0.76484 & 0.031160 \\ \hline
	\end{tabular}
	$$

	\bigbreak
	We find the error term $\mid \frac{h}{2}f''(\xi)\mid$ from the following formula
	$$
		f'(x_0) = \frac{f(x_0+h) - f(x_0)}{h} - \frac{h}{2}f''(\xi)
	$$
	This error bound will be the same for forward and backward difference.
	We have $f''(x) = -\sin x$  and $h = 0.1$ in all cases, thus our error is given by
	$\mid \frac{0.1}{2}\sin \xi \mid$ where $\xi$ is a value within the bounds of our difference.
	For each, we therefore want to maximize the value $\sin \xi$.
	
	\bigbreak
	\textbf{$f'(0.5)$:} $max_{0.5,0.6}\ \sin \xi = 0.564642 \Rightarrow \upepsilon = 
	\frac{0.1}{2}0.564642 = 0.028232$ \\
	Thus we have an error bound of $0.02832$ which is just a bit over our actual error 
	of $0.025580$.

	\bigbreak
	\textbf{$f'(0.6)$:} $max_{0.6,0.7}\ \sin \xi = 0.644218 \Rightarrow \upepsilon = 
	\frac{0.1}{2}0.644218 = 0.03221$ \\
	Thus we have an error bound of $0.03221$ a little bit worse than before. Our actual
	error of $0.029340$ fits nicely within this bound.

	\bigbreak
	\textbf{$f'(0.7)$:} $max_{0.6,0.7}\ \sin \xi = 0.644218 \Rightarrow \upepsilon = 
	\frac{0.1}{2}0.644218 = 0.03221$ \\
	Since we are taking the backward difference, we come to the same error bound
	as the forward difference for $f'(0.6)$, that is $0.03221$. Our actual error
	for this value is the worst at $0.031160$ but still within the error bounds
	like we would expect.

\part

	Let $f(x) = \sin(x)$, and $x = 3$. \\
	We have $f'(x) = \cos(x)$, thus $f'(3) = \cos(3) = -0.9899924966$. \\

	$$
	\begin{tabular}{|c|c|c|c|}
		\hline
		n & $f'_n(x)$ & $\mid f'_n(x) - f'(x) \mid$ \\ \hline
		1 & -0.995393456265767 & 0.005400959665322 \\ \hline
		2 & -0.990681590968298 & 0.000689094367853 \\ \hline
		3 & -0.990062891599724 & 0.000070394999279 \\ \hline
		4 & -0.989999550952969 & 0.000007054352524 \\ \hline
		5 & -0.989993202188399 & 0.000000705587954 \\ \hline
		6 & -0.989992567285158 & 0.000000070684713 \\ \hline
		7 & -0.989992501865267 & 0.000000005264821 \\ \hline
		8 & -0.989992490763036 & 0.000000005837409 \\ \hline
		9 & -0.989992560151975 & 0.000000063551530 \\ \hline
		10 & -0.989992532396400 & 0.000000035795954 \\ \hline
	\end{tabular}
	$$

	In general, for larger values of $n$ we would expect the error between
	our approximation and the actual derivative to go down. However, if we 
	in the above example we can see the error actually increase from $n=7$ to
	$n=8$ and from $n=8$ to $n=9$. We do see a decrease again from $n=9$ to $n=10$
	however we still don't have the precision found at $n=7$ right before our estimates
	started to get worse.

%% Problem 2
\problem{} (60 points)

	\begin{enumerate}
		\item Approximate the following integral using the Trapezoidal rule, find a bound for the
			error using error formula and compare this to the actual error:
			$$
				\int_{0.5}^{1} x^4 dx.
			$$
		\item Repeat part (a) using Simpson's rule.
		\item Repeat part (a) using Composite Trapezoidal rule with $n=4$.
		\item Repeat part (a) using Composite Simpson rule with $n=4$.
		\item Write a code to implement part (c) and (d) in MATLAB.
		\item Write a function $v = $ CompositeTrapezoidalRule($f,a,b,n$) to implement
			Composite trapezoidal rule with a given $n$ for
			$$
				\int_a^b f(x) dx,
			$$
			verify your code with part (c).
	\end{enumerate}

\solution

	\part
		By the trapezoidal rule, our approximation is given by $\frac{h}{2}[f(x_0) + f(x_1)]$.
		Where $x_0 = 0.5, x_1 = 1$ gives us $f(x_0) = 0.0625, f(x_1) = 1.0$.
		The estimate is given by $1.0625\frac{0.5}{2} = 0.26562$. The actual value
		of the integral is $0.19375$, thus our actual error is
		$\mid 0.26562 - 0.19375\mid = 0.071875$.

		\bigbreak
		The trapezoidal rule has an error term of $\frac{h^3}{12}f''(\xi)$. Where
		$f''(\xi) = 12\xi^2$, maximizing this on the bounds $(0.5,1)$ gives us
		$12$, thus our error bound $= \frac{0.5^3}{12}12 = 0.5^3 = 0.125$.
		Note that our actual error $0.071875$ is within this error bound.

	\part

		First we compute the values needed by Simpson's rule. We have $a=0.5, b=1.0$ 
		thus $h = (1.0-0.5)/2 = 0.25, x_0 = a = 0.5, x_1 = a + h = 0.75, x_2 = 1.0$.
		Simpson's rule is given by $\frac{h}{3}[f(x_0) + 4f(x_1) + f(x_2)]$, filling
		in our values gives us $\frac{0.25}{3}[0.5^4 + 4\times 0.75^4 + 1^4] = 
		0.08333[0.0625 + 1.2656 + 1] = 0.08333[2.3281] = 0.194$.
		Given the actual value of the integral as $0.19375$ our actual error is 
		then $\mid0.194 - 0.19375\mid = 0.00025057$.

		\bigbreak
		Simpson's rule has an error term of $\frac{h^5}{90}f^{(4)}(\xi)$. Where
		$f^{(4)}(\xi) = 24$. Our error bound is then $24\frac{0.25^5}{90} = 0.00026042$.
		Our actual error sneaks just under this error bound.

	\part
		
		For Composite Trapezoidal Rule with $n=4$, we have $h = (1.0-0.5)/4 = 0.125,
		x_i = [0.5, 0.625, 0.75, 0.875, 1.0], i \in [0,1,\ldots,4]$. The Composite
		Trapezoidal Approximation is given by $\frac{h}{2}[f(a) + 2\Sigma_{j=1}^{n-1}
		f(x_j) + f(b)]$. Plugging in our values gives us
		\begin{align*}
			\int_{0.5}^{1} x^4 dx \approx
			&\frac{0.125}{2}[f(0.5) + 2(f(0.625) + f(0.75) + f(0.875)) + f(1.0)] \\
			= & \frac{0.125}{2}[0.0625 + 2(1.0557) + 1] \\
			= & \frac{0.125}{2}[3.1739] \\
			= & 0.19837
		\end{align*}
		With an actual value of $0.19375$ our actual error is
		$0.19837 - 0.19375 = 0.0046188$.

		\bigbreak
		The error term is given by $\frac{b-a}{12}h^2 f''(\mu)$. From part (a) we know
		we can maximize $f''(\mu)$ as $12$. Thus the error bound is 
		$12 \frac{1-0.5}{12}0.125^2 = 0.0078125$.
		Again, our actual is nicely within this bound.

	\part

		For Composite Simpson's Rule with $n=4$, we have $h=(1.0-0.5)/4=0.125,
		x_i = [0.5, 0.625, 0.75, 0.875, 1.0], i \in [0,1,\ldots,4]$, these are the
		same values obtained from Composite Trapezoidal Rule. The desired 
		approximation is given by 
		$\frac{h}{3}[f(a) + 2\Sigma_{j=1}^{(n/2)-1}f(x_{2j} + 4\Sigma_{j=1}^{n/2}f(x_{2j-1}) + f(b)]$. 
		Plugging in our values gives us
		\begin{align*}
			\int_{0.5}^{1} x^4 dx \approx
				& \frac{0.125}{3}[f(0.5) + 2(f(0.75)) +
				4(f(0.625) + f(0.875)) + f(1.0)] \\ 
			= & \frac{0.125}{3}[0.625 + 2(0.31641) + 
				4(0.73877) + 1.0] \\
			= & \frac{0.125}{3}[4.6504] \\
			= & 0.193766276 \\
		\end{align*}
		With an actual value of $0.19375$ our actual error is
		$0.193766276 - 0.19375 = 0.0000162760416666519$.

		\bigbreak
		The error term is given by $\frac{b-a}{180}h^4f^{(4)}(\mu)$. From part (b) we
		know we $f^{(4)}(\mu) = 24$. Putting it all together yields 
		$24\frac{1.0-0.5}{180}0.125^4 = 0.0000162760416666667$. Here we have a pretty
		accurate error bound and our actual value is falling just under it, though at this
		number of significant digits we might expect computer precision to be affecting our results.

		\part

		For CompositeTrapezoidalRule see my answer to part (f).
		\textcolor[RGB]{240,240,240}{\rule{\textwidth}{0.5pt}}\bigbreak

		\inputminted{octave}{CompositeSimpsonsRule.m}
		Running 
		\begin{minted}{octave}
		CompositeSimpsonsRule(@(x)x.^4, 0.5, 1, 4)
		\end{minted}
		produces a result of $0.19377$ which is the same I found in part (d).

		\part

		\inputminted{octave}{CompositeTrapezoidalRule.m}
		Running 
		\begin{minted}{octave}
		CompositeTrapezoidalRule(@(x)x.^4, 0.5, 1, 4)
		\end{minted}
		produces a result
		of $0.19830$ which is roughly the same I found in part (c). There is some variation
		at later significant digits, but I would attribute to truncation when computing by hand.

%% Problem 3
\problem{} (20 points)

	\begin{enumerate}
		\item Show that the following quadrature formula has a degree of precision equal to 3,
			$$
				\int_{-1}^{1} f(x) dx = f(-\frac{\sqrt{3}}{3}) + f(\frac{\sqrt{3}}{3}).
			$$
		\item Approximate the following integral using Gaussian quadrature with $n= 2,3,5$,
			$$
				\int_1^{1.5} x^2 \ln x dx
			$$
	\end{enumerate}

\solution

	\part

	\begin{proof} \\
		If the quadrature formula has a degree of precision equal to 3, then
		$x^k$ must be exact for each $k=0,1,2,3$. Further 3 must be the largest integer
		for which this is true. Thus if it is in fact true for $k=0,1,2,3$ (as we will
		show), it then must not be true for $k=4$ else the quadrature formula
		would have a degree of precision equal to 4.

		\bigbreak
		\textbf{k=0:}
		$\int_{-1}^{1} x^0 dx = \int_{-1}^{1} dx = 2$, further 
		$(-\frac{\sqrt{3}}{3})^0 + \frac{\sqrt{3}}{3}) = 1 + 1 = 2$. Therefore we
		have at least a quadrature with degree of precision 0.

		\bigbreak
		\textbf{k=1:}
		$\int_{-1}^{1} x dx = 0$, further 
		$(-\frac{\sqrt{3}}{3}) + \frac{\sqrt{3}}{3}) = 0 \Rightarrow$ our degree of
		precision is at least 1.
		
		\bigbreak
		\textbf{k=2:}
		$\int_{-1}^{1} x^2 dx = \frac{2}{3}$, further 
		$(-\frac{\sqrt{3}}{3})^2 + \frac{\sqrt{3}}{3})^2 = \frac{1}{3} + \frac{1}{3}= \frac{2}{3} 
		\Rightarrow$ our degree of precision is at least 2.

		\bigbreak
		\textbf{k=3:}
		$\int_{-1}^{1} x^3 dx = 0$, further 
		$(-\frac{\sqrt{3}}{3})^3 + \frac{\sqrt{3}}{3})^3 =
		-(\frac{\sqrt{3}}{3})^3 + \frac{\sqrt{3}}{3})^3 = 0
		\Rightarrow$ our degree of precision is at least 3.
		
		\bigbreak
		\textbf{k=4:}
		$\int_{-1}^{1} x^2 dx = \frac{4}{10}$, further 
		$(-\frac{\sqrt{3}}{3})^4 + \frac{\sqrt{3}}{3})^4 = 
		2(\frac{\sqrt{3}}{3})^4 = 2\frac{9}{81} = 2\frac{1}{9} = \frac{2}{9}$. But as
		$\frac{4}{10} \neq \frac{2}{9} \Rightarrow$ our degree of precision is less than 4.

		\bigbreak
		Note that by a generalization of the argument provided for $k=3$ we know that
		the quadrature formula will be true for all odd $k$, but for a degree of precision
		of $k$ we need it to be true for all integers less than or equal to $k$. Thus as the formula is 
		not exact at $k=4$ and is exact for $k=0,1,2,3$ we know that we must have a degree
		of precision equal to 3.

	\end{proof}

	\clearpage
	\part

		Before we begin, we must first transform the given integral on the bounds 
		$1,1.5$ to the bounds $-1,1$ through the following transformation
		\begin{align*}
			\int_1^{1.5} f(x)dx 
			&= \int_{-1}^1 f\Big(\frac{(1.5-1)t + 1 + 1.5}{2}\Big)\frac{(1.5-1)}{2}dt \\
			&= \int_{-1}^1 f\Big(\frac{0.5t + 2.5}{2}\Big)0.25dt \\
			&= \int_{-1}^1 0.25 f(0.25t + 1.25)dt
		\end{align*}
		Thus define $g(x) = 0.25 f(0.25x + 1.25)$.
		Therefore the approximation via Gaussian
		quadrature is given by $\Sigma_{i=1}^{n}c_i g(x_i)$.
		Note that the actual value of the integral we are approximating is $0.192259$.

		\bigbreak
		\textbf{n=2:} \\
		From the table in the lecture notes $c = [1.0, 1.0]$.
		The points we would like to consider are $x_i = [-1, 1]$, plugging these
		into the formula above gives us
		\begin{align*}
			\int_{1}^{1.5}x^2\ln x dx &\approx
				c_1 g(x_1) + c_2 g(x_2) \\
			&= g(-1) + g(1) \\
			&= 0.22807 \\
		\end{align*}
		This is an error of $0.035811$.

		\bigbreak
		\textbf{n=3:} \\
		From the table in the lecture notes $c = [\frac{5}{9}, \frac{8}{9}, \frac{5}{9}]$.
		The points we would like to consider are $x_i = [-1, 0, 1]$, plugging these
		into the formula above gives us
		\begin{align*}
			\int_{1}^{1.5}x^2\ln x dx &\approx
				c_1 g(x_1) + c_2 g(x_2) + c_3 g(x_3) \\
			&= \frac{5}{9}g(-1) + \frac{8}{9}g(0) + \frac{5}{9}g(1)\\
			&= 0.0 + 0.07748 + 1.2671 \\
			&= 0.20419 \\
		\end{align*}
		This is an error of $0.011931$.

		\clearpage
		\textbf{n=5:} \\
		From the table in the lecture notes 
		$c = [0.2369268850, 0.4786286705, 0.5688888889, 0.4786286705, 0.2369268850]$.
		The points we would like to consider are $x_i = [-1, -0.5, 0, 0.5, 1]$, plugging these
		into the formula above gives us
		\begin{align*}
			\int_{1}^{1.5}x^2\ln x dx &\approx
				c_1 g(x_1) + c_2 g(x_2) + c_3 g(x_3) + c_4 g(x_4) + c_5 g(x_5) \\
			&= 0.2369268850g(-1.0) + 0.4786286705g(-0.5) + \\
			&\ \ 0.5688888889g(0.0) + 0.4786286705g(0.5) + 0.2369268850g(1.0) \\
			&= 0.000000 + 0.017837 + 0.049587 + 0.072043 + 0.054037 \\
			&= 0.19350
		\end{align*}
		This is an error of $0.0012410$.
		\bigbreak
		We can see that as we increase $n$ we get a more accurate approximation of the
		actual integral.

%% Problem 4
\problem{} (20 points)

	\begin{enumerate}
		\item Use Composite Simpson rule with $n=4$ to approximate the following integral
			$$
				\int_0^1 x^{-1/4} \sin xdx
			$$
		\item Use Composite Simpson rule with $n=6$ to approximate the following integral
			$$
				\int_1^{\infty} \frac{\cos x}{x^3} dx
			$$
	\end{enumerate}

\solution

%% Problem 5
\problem{} (30 points)

	\begin{enumerate}
		\item Use the Gaussian Elimination with Backward substitution to solve the following
			linear system (must show intermediate steps).
			\begin{alignat*}{2}
				2x_1 &- 1.5x_2 &+ 3x_3 &= 1 \\
				-x_1 &         &+ 2x_3 &= 3 \\
				4x_1 &- 4.5x_2 &+ 5x_3 &= 1 \\
			\end{alignat*}
		\item Repeat (a) using the Gaussian Elimination with partial pivoting and Backward
			substitution (must show intermediate steps).
		\item Repeat (a) using Gaussian Elimination with scaled partial pivoting and 
			Backward substitution (must show intermediate steps).
	\end{enumerate}

\solution

%% Problem 6
\problem{} (20 points)

	\begin{enumerate}
		\item Solve the following linear system with forward and backward substitution,
			$$
			\begin{bmatrix}
				1 & 0 & 0 \\
				2 & 1 & 0 \\
				-1 & 0 & 1 \\
			\end{bmatrix}
			\begin{bmatrix}
				2 & 3 & -1 \\
				0 & -2 & 1 \\
				0 & 0 & 3 \\
			\end{bmatrix}
			\begin{bmatrix}
				x_1 \\ x_2 \\ x_3 \\
			\end{bmatrix} = 
			\begin{bmatrix}
				2 \\ -1 \\ 1
			\end{bmatrix}
			$$
		\item Use the LU factorization (no permutation) with forward and backward
			substitution to solve the following linear system (Use the sample MATLAB
			codes posted on classpage as discussed in class),
			\begin{alignat*}{2}
				2x_1 &+ 3x_2 &- x_3 &= 2 \\
				4x_1 &+ 4x_2 &- x_3 &= -1 \\
				-2x_1 &- 3x_2 &+ 4x_3 &= 1
			\end{alignat*}
	\end{enumerate}

\solution

%% Problem 7
\problem{} (20 points)

	\begin{enumerate}
		\item Use the $A = LDL^t$ factorization to solve the following linear system,
			\begin{alignat*}{2}
				2x_1 &- x_2 & &= 2 \\
				-x_1 &+ 2x_2 &- x_3 &= -1 \\
				&- x_2 &+ 2x_3 &= 1
			\end{alignat*}
		\item Repeat (a) with Cholesky factorization $A = LL^t$.
	\end{enumerate}

\solution

\end{document}
