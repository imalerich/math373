\documentclass[10pt]{jhwhw}
\author{Ian Malerich}
\title{Math S 373: Homework 1}
\usepackage{amssymb, amsfonts, amsmath, mathtools, graphicx, breqn, soul}
\usepackage{minted, subfig, float, scrextend, setspace, amsthm}
\usemintedstyle{friendly}

\begin{document}
\raggedright

%% Problem 1
\problem{} (10 points)
	Find the least squares polynomial of degree 2 for the data in the following
	table. Compute the error E. Graph the data and the polynomial.
	\bigbreak
	\begin{tabular}{ccccccc}
		\hline
		$x_i$ & 1.0 & 1.1 & 1.3 & 1.5 & 1.9 & 2.1 \\
		$y_i$ & 1.84 & 1.96 & 2.21 & 2.45 & 2.94 & 3.18 \\
		\hline
	\end{tabular}

\solution

	A = 
	\left[\begin{array}{ccc}
		1 & 1 & 1 \\
		1 & 1.1 & 1.1^2 \\
		1 & 1.3 & 1.3^2 \\
		1 & 1.5 & 1.5^2 \\
		1 & 1.9 & 1.9^2 \\
		1 & 2.1 & 2.1^2 \\
	\end{array} \right]
	b = 
	\left[\begin{array}{c}
		1.84 \\ 1.96 \\ 2.21 \\ 2.45 \\ 2.94 \\ 3.18 \\
	\end{array} \right]

	\bigbreak
	$A^TAa = A^Tb \Rightarrow (A^TA)^{-1}(A^TA)a = (A^TA)^{-1}A^Tb \Rightarrow
	a = (A^TA)^{-1}A^Tb$

	\begin{align*}
	a &= 
	\left[\begin{array}{ccc}
		66.841 & -90.719 & 28.748 \\
		-90.719 & 124.544 & -39.811 \\
		28.748 & -39.811 & 12.832 \\
	\end{array} \right]
	\left[\begin{array}{ccc}
		1 & 1 & 1 & 1 & 1 & 1 \\
		1 & 1.1 & 1.3 & 1.5 & 1.9 & 2.1 \\
		1 & 1.21 & 1.69 & 2.25 & 3.61 & 4.41 \\
	\end{array} \right]
	\left[\begin{array}{c}
		1.84 \\ 1.96 \\ 2.21 \\ 2.45 \\ 2.94 \\ 3.18 \\
	\end{array} \right]
	\\ &= 
	\left[\begin{array}{cccccc}
		4.86939 & 1.83451 & -2.51039 & -4.55547 & -1.74620 & 3.10816 \\
		-5.98639 & -1.89239 & 3.90693 & 6.52134 & 2.19542 & -4.74490 \\
		1.76871 & 0.48237 & -1.32035 & -2.09647 & -0.56895 & 1.773469 \\
	\end{array} \right]
	\left[\begin{array}{c}
		1.84 \\ 1.96 \\ 2.21 \\ 2.45 \\ 2.94 \\ 3.18 \\
	\end{array} \right]
	\\ &= 
	\left[\begin{array}{c}
		0.596581 \\ 1.253293 \\ -0.010853 \\
	\end{array} \right]
	\end{align*}

	This gives us the interpolating polynomial as
	$
		P_2(x) = -0.010853x^2 + 1.253293x + 0.596581
	$.

	Evaluating our initial points using our polynomial produces the results...
	\bigbreak
	\begin{tabular}{ccccccc}
		\hline
		$x_i$ & 1.0 & 1.1 & 1.3 & 1.5 & 1.9 & 2.1 \\
		$y_i$ & 1.84 & 1.96 & 2.21 & 2.45 & 2.94 & 3.18 \\
		$P(x_i)$ & 1.839 & 1.9621 & 2.2075 & 2.4521 & 2.9387 & 3.1806 \\
		$|P(x_i)-y_i|$ & -9.79e-4 & 2.0712e-3 & 2.4797e-3 & 2.1012e-3 & 1.3416e-3 & 6.3457e-4 \\
		\hline
	\end{tabular} \bigbreak
	Taking the sum of the differences finds the total error across all interpolating
	points as $E = 6.69\times 10^{-6}$.

	\bigbreak
	We can see that the least squares fit finds a polynomial which is 
	almost linear, but not quite. From both the graph below, and the error above,
	the polynomial is clearly doing a good job (though not perfect) of fitting to
	the points.
	\includegraphics[scale=0.75]{p1}

%% Problem 2
\problem{} (20 points)

	\begin{enumerate}
		\item Find the least squares polynomial of degree 2 for the following function
			$f(x)$ on the indicated interval
			$$
				f(x) = e^x,\ [-1,1]
			$$
			Compute the error E. Graph the function and the polynomial.
		\item Repeat (a) with Legendre Polynomials.
	\end{enumerate}

\solution

	\part
	I will be using the $x=(-1,0,1)$ as interpolating points to 
	find my least squares polynomial, thus we want to use the following matrices
	in our system.

	\bigbreak
	$
	A = 
	\left[\begin{array}{ccc}
		1 & -1 & 1 \\
		1 & 0 & 0 \\
		1 & 1 & 1 \\
	\end{array} \right]
	b = 
	\left[\begin{array}{c}
		\frac{1}{e} & 1 & e \\
	\end{array} \right]
	$
	\bigbreak
	$A^TAa = A^Tb \Rightarrow (A^TA)^{-1}(A^TA)a = (A^TA)^{-1}A^Tb \Rightarrow
	a = (A^TA)^{-1}A^Tb$

	\bigbreak
	\begin{align*}
		a &= 
		\left[\begin{array}{ccc}
			1 & 0 & -1 \\
			0 & 0.5 & 0 \\
			-1 & 0 & 1.5 \\
		\end{array} \right]
		\left[\begin{array}{ccc}
			1 & 1 & 1 \\
			-1 & 0 & 1 \\
			1 & 0 & 1 \\
		\end{array} \right]
		\left[\begin{array}{c}
			\frac{1}{e} & 1 & e \\
		\end{array} \right]
		\\ &= 
		\left[\begin{array}{ccc}
			0 & 1 & 0 \\
			-0.5 & 0 & 0.5 \\
			0.5 & -1 & 0.5 \\
		\end{array} \right]
		\left[\begin{array}{c}
			\frac{1}{e} & 1 & e \\
		\end{array} \right]
		\\ &=
		\left[\begin{array}{c}
			1 \\ 1.7520 \\ 0.54308
		\end{array} \right]
	\end{align*}
	\bigbreak
	This gives us the interpolating polynomial $P_2(x) = 0.54308x^2 + 1.17520x + 1$.
	Evaluating our initial points using our polynomial produces the results\ldots

	\bigbreak
	\begin{tabular}{cccc}
		\hline
		$x_i$ & -1.0 & 0 & 1.0 \\
		$y_i$ & $\frac{1}{e}$ & 1 & e \\
		$P(x_i)$ & 0.36788 & 1.0 & 2.71828 \\
		$|P(x_i)-y_i|$ & 5.5883e-7 & 0 & 1.8285e-6 \\
		\hline
	\end{tabular}
	\bigbreak
	Taking the sum of the differences finds the total error across all interpolating
	points as $E = 2.3873\times 10^{-6}$.
	If we look over the range $[-1, 1]$ and want to compute the total error 
	(not just the interpolating points), we simply compute
	$\int_{-1}^1 |P(x) - e^x|dx = 0.0890387$.
	
	\bigbreak
	From the graph below, we can see that the interpolating polynomial closely matches,
	but diverges slightly from $e^x$. Notably, the polynomial switches from 
	underestimating to overestimating after crossing $x=0$.
	\includegraphics[scale=0.75]{p2}

	\part

	$b = 
	\left[\begin{array}{c}
		\int_{-1}^1 e^x P_0(x) dx \\
		\int_{-1}^1 e^x P_1(x) dx \\
		\int_{-1}^1 e^x P_2(x) dx \\
	\end{array} \right] =
	\left[\begin{array}{c}
		\int_{-1}^1 e^x dx \\
		\int_{-1}^1 e^x x dx \\
		\int_{-1}^1 e^x \frac{1}{2}(3x^2-1) dx \\
	\end{array} \right] =
	\left[\begin{array}{c}
		e - \frac{1}{e} \\
		\frac{2}{e} \\
		0.143126 \\
	\end{array} \right] =
	\left[\begin{array}{c}
		2.3504 \\
		0.73576 \\
		0.143126 \\
	\end{array} \right]
	$
	\bigbreak
	$d = 
	\left[\begin{array}{c}
		\frac{2(1)-1}{2}b_1 \\
		\frac{2(2)-1}{2}b_2 \\
		\frac{2(3)-1}{2}b_3 \\
	\end{array} \right] =
	\left[\begin{array}{c}
		\frac{1}{2}\times 2.3504 \\
		\frac{3}{2}\times 0.73576 \\
		\frac{5}{2}\times 0.143126 \\
	\end{array} \right] =
	\left[\begin{array}{c}
		1.1752 \\
		1.1036 \\
		0.35781 \\
	\end{array} \right]
	$
	\bigbreak
	The Legendre Polynomial Approximation is then given by\ldots
	\begin{align*}
		1.1752P_0 + 1.1036P_1 + 0.35781P_2 &=
		1.1752 + 1.1036x + 0.35781\frac{1}{2}(3x^2-1)  \\
		&= 0.536715x^2 + 1.1036x + 0.996295 \\
	\end{align*}

	\clearpage
	We can compute error over the bounds by taking $\integral_{-1}^1 |L_2(x) - e^x|dx
	= 0.0458856$. Recall the error for least squares was $0.0890387$, so we have 
	approximately halved our error despite both polynomials having degree 2. Note that
	we may have found a better least squares polynomial if we had used different
	interpolating points, these were not necessary for the Legendre approximation.

	\bigbreak
	Below we can visually confirm that this polynomial is doing a decent job
	of approximating $e^x$ on our given bounds. It appears to be more consistent
	and accurate than the least squares polynomial 
	but with greater error towards the end points.
	\includegraphics[scale=0.75]{p2b}

%% Problem 3
\problem{} (20 points)

	\begin{enumerate}
		\item Use the zeros of $\overline{T}_4$ to construct an interpolating polynomial of degree
			3 for the following function on the interval $[-1,1]$:
			$$
				f(x) = e^x
			$$
			Find a bound for the maximum error of the approximation.
		\item Repeat (a) on interval $[0, 2]$.
	\end{enumerate}

\solution

	\part

	First, note that we will have 4 zeros of $T_4$ given by $\cos(\frac{2k-1}{2n}\pi)$.
	These are the values $x_0 = \{0.92388, 0.38268, -0.38268, -0.92366\}$.
	Evaluating our function at these points produces $f(x_0) = \{2.51904, 1.46621, 0.68203, 0.39698\}$.
	Next we set up our linear equation to find the coefficients of our polynomial.

	\bigbreak
	\left[\begin{array}{cccc}
		0.92388^3 & 0.92388^2 & 0.92388 & 1 \\
		0.38268^3 & 0.38268^2 & 0.38268 & 1 \\
		-0.38268^3 & -0.38268^2 & -0.38268 & 1 \\
		-0.92388^3 & -0.92388^2 & -0.92388 & 1 \\
	\end{array} \right]
	\left[\begin{array}{c}
		a \\ b \\ c \\ d \\
	\end{array} \right] =
	\left[\begin{array}{c}
		2.51904 \\ 1.46621 \\ 0.68203 \\ 0.39696 \\
	\end{array} \right]

	\bigbreak
	Solving this yields
	\bigbreak
	\left[\begin{array}{c}
		a \\ b \\ c \\ d \\
	\end{array} \right] =
	\left[\begin{array}{c}
		0.17518 \\ 0.54290 \\ 0.99893 \\ 0.99462 \\
	\end{array} \right] \bigbreak
	Which produces the 4th degree polynomial $0.17518x^3 + 0.54290x^2 + 0.99893x + 0.99462$. \\
	To find an upper bound for the error on the bounds I simply need to solve
	$$
		\text{max}_{-1,1}(|0.17518x^3 + 0.54290x^2 + 0.99893x + 0.99462 - e^x|)
	$$
	This occurs at $x=1$ which evaluates to an error of $e - \frac{271163}{100000} \approx 0.00665183$.

	\clearpage
	Once printed out you may not be able to see $e^x$ below the interpolating polynomial,
	clearly this method is doing a pretty decent job of approximating on the given bounds.
	\includegraphics[scale=0.75]{p3}

	\part
	We can follow the same process as part a, but instead use the 4 zeros given by 
	$1 + \cos(\frac{2k-1}{2n}\pi)$. These are the zeros 
	$x_0 = \{1.923880, 1.382683, 0.617317, 0.076120\}$. Evaluating our function at these
	points produces $f(x_0) = \{6.8475, 3.9856, 1.8539, 1.0791\}$. Again, we set up
	and solve our linear equation to find the coefficients of our polynomial.

	\bigbreak
	\left[\begin{array}{cccc}
		1.92388^3 & 1.92388^2 & 1.92388 & 1 \\
		1.38268^3 & 1.38268^2 & 1.38268 & 1 \\
		0.617317^3 & 0.617317^2 & 0.617317 & 1 \\
		0.076120^3 & 0.076120^2 & 0.076120 & 1 \\
	\end{array} \right]
	\left[\begin{array}{c}
		a \\ b \\ c \\ d \\
	\end{array} \right] =
	\left[\begin{array}{c}
		6.8475 \\ 3.9856 \\ 1.8539 \\ 1.0791 \\
	\end{array} \right]

	\bigbreak
	Solving this yields
	\bigbreak
	\left[\begin{array}{c}
		a \\  b \\  c \\ d \\
	\end{array} \right] =
	\left[\begin{array}{c}
		0.476177 \\ 0.047226 \\ 1.192398 \\ 0.987843 \\
	\end{array} \right] \bigbreak

	\bigbreak
	Which gives us the polynomial $0.476177x^3 + 0.047226x + 1.192398x + 0.987843$.
	To find an upper bound for the error on the bounds I simply need to solve
	$$
		\text{max}_{0,2}(|0.476177x^3 + 0.047226x + 1.192398x + 0.987843 - e^x|)
	$$
	This occurs at $x=2$ which evaluates to an error of $e^2 - \frac{1474191}{200000} \approx 0.018101$.
	We can see that as $e^x$ starts to blow up, our polynomial will obviously start to perform worse
	relative to the bounds of part (a).

	\includegraphics[scale=0.75]{p3b}

%% Problem 4
\problem{} (20 points)
	Find all the Chebyshev rational approximations of degree 2 for $f(x) = e^{-x}$.
	Graph the function and the polynomial. (Use MATLAB routines on classpage)

\solution

	TODO

%% Problem 5
\problem{} (30 points) (Use MATLAB routines on classpage)

	\begin{enumerate}
		\item Find the continous least squares trigonometric polynomial $S_3(x)$ for
			$f(x) = e^x$ on $[-\pi,\pi]$.
		\item Find the discrete least squares trigonometric polynomials $S_n(x)$ for
			$f(x) = e^x$ on $[-\pi, \pi]$ with $n=3, m=6$.
		\item Find the trigonometric interpolating polynomial $S_n(x)$ for 
			$f(x) = e^x$ on $[-\pi, \pi]$ with $n=8$.
	\end{enumerate}

\solution

	TODO

\end{document}
