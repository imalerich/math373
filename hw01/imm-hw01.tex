\documentclass[10pt]{jhwhw}
\author{Ian Malerich}
\title{Math S 373: Homework 1}
\usepackage{amssymb, amsfonts, amsmath, mathtools, graphicx, breqn}
\usepackage{minted, subfig, float, scrextend, setspace}
\usemintedstyle{friendly}

\begin{document}
\raggedright

%% Problem 1
\problem{}

	\begin{enumerate}
		\item Show that $x-(\ln x)^x = 0$ has at least one solution on $[4,5]$.
		\item Use the Intermediate Value Theorem and Rolle's Theorem to show that the graph
			of $f(x) = x^3 + 2x + k$ crosses the x-axis exactly once, regardless of the value of the
			constant k.
	\end{enumerate}

\solution

\part

	\begin{proof}
		Note that the function is continous, in particular, it is continous over $[4,5]$.
		Further as we have that $f(4) = 4 - ln(4)^4 = 0.306638422$ and $f(5) = 5 - ln(5)^5 = -5.798691578$, and
		$-5.79869 < 0 < 0.306638422$, then by the Intermediate Value Theorem, $\exists x\in[4,5] \text{ such that }
		f(x) = 0$.
	\end{proof}

\part

	\begin{proof}
		Proof goes here.
	\end{proof}

%% Problem 2
\problem{}

	Let $f(x) = 2x\cos (2x) - (x-2)^2 \text{ and } x_0 = 0$.
	\begin{enumerate}
		\item Find the third Taylor polynomial $P_3(x)$, and use it to approximate $f(0.4)$
		\item Use the error formula in Taylor's Theorem to find an upper bound for the error
			$|f(0.4) - P_3(0.4)|$. Compute the actual error.
	\end{enumerate}

\solution

\part

	$P_3(x) = f(x_0) + f'(x_0)(x-x_0) + \frac{f''(x_0)}{2}(x-x_0)^2 + \frac{f^{(3)}(x_0)}{6}(x-x_0)^3$
	\begin{align*}
		f(0) = 2\times 0\cos (0) - (-2)^2 = -&4 \\
		f'(0) = -2x - 4x\sin (2x) + 2\cos (2x) + 4 = 2\cos 0 + 4 = 2 + 4 =\ &6 \\
		f''(0) = -2(4\sin (2x) + 4x\cos (2x) + 1) = -2(0 + 0 + 1) = -&2 \\
		f^{(3)}(0) = 8(2x\sin (2x) - 3\cos (2x)) = 8(0 - 3) = -&24
	\end{align*}

	Thus, the third Taylor polynomial about $x_0 = 0$ is given by
	$$
		P_3(x) = -4 + 6x + -x^2 + -4x^3
	$$
	Then
	$$
		f(0.4) \approx P_3(0.4) = -2.016
	$$

\part

	We have that $f^{(4)}(x) = 32(2\sin (2x) + x\cos (2x))$. \\
	The error is then given by the following (for some $c_x \in (0,0.4)$):
	$$
		\mathcal{E} (x) = \frac{f^{(4)}(c_x)}{24}x^4
	$$
	Thus, to find an upper bound of the error, we must find an upper bound for $f^{(4)}(c_x)$.
	This is found at $f^{(4)}(0.4) = 54.8286$. Giving us an error of 
	$\frac{54.8286}{24}(0.4)^4 = 0.05848384$.

	Actual error is given by $|f(0.4) - P_3(0.4)| = |-2.00263 - -2.016| = 0.0133654$.

%% Problem 3
\problem{}

	Let $f \in C[a,b]$ be a function whose derivative exists on $(a,b)$. Suppose $f$ is to be evaluated
	at $x_0$ in $(a,b)$, but instead of computing the actual value $f(x_0)$, the approximation value,
	$\widetilde{f}(x_0)$, is the actual value of $f$ at $x_0 + \epsilon$, that is 
	$\widetilde{f}(x_0) = f(x_0 + \epsilon)$.

	\begin{enumerate}
		\item Use the Mean Value Theorem to estimate the absolute error $|f(x_0) - \widetilde{f}(x_0)|$
			and the relative error $|f(x_0) - \widetilde{f}(x_0)|/|f(x_0)|$, assuming $f(x_0) \neq 0$.
		\item If $\epsilon = 5 \times 10^{-6}$ and $x_0=1$, find bounds for the absolute and relative errors for
			\begin{enumerate}
				\item $f(x) = e^x$
				\item $f(x) = \sin x$
			\end{enumerate}
	\end{enumerate}

\solution

\part 

	For absolute error, we have $|f(x_0) - \widetilde{f}(x_0)| = |f(x_0) - f(x_0+\epsilon)|$ by
	the mean value theorem it follows that
	$|f(x_0) - f(x_0+\epsilon)| = |f'(c)\epsilon|, c \in (x_0, x_0 + \epsilon)$. \\
	Therefore, our bound for absolute error is given by
	$\epsilon\times\sup|\{f'(c) : c\in (x_0,x_0+\epsilon)\}|$. \\
	Similarly, the bound for relative error is thus given by
	$\epsilon\times\sup|\{f'(c) : c\in (x_0,x_0+\epsilon)\}| / |f(x_0)|$.

\part

\textbf{(i)} $f(x) = e^x$

	To find absolute error by the formula given in part (a), we simply need to find the maximum
	derivative of $f(x)$ over $(1, 1 + 5\times10^{-6})$. As $e^x$ is non-decreasing, and $e^x' = e^x$
	the maximum is given by $e^{(1+5\times 10^{-6})} \approx 2.7182954199$ we then multiply this by $\epsilon$ to
	get the absolute error = $e^{1+5\times 10^{-6}}5\times 10^{-6} \approx 0.000013591477$.
	\bigbreak
	Relative error is then $(e^{1+5\times 10^{-6}}5\times 10^{-6})/(e^1) \approx 5.000025\times 10^{-6}$.

\bigbreak
\textbf{(ii)} $f(x) = \sin x$

	$f'(x) = \cos(x)$, thus to determine bound for absolute error, I need the maximum value
	of $\cos(x)$ over $(1, 1+5\times 10^{-6})$. Between 0 and $\pi/2\, \cos(x)$ is positive
	and decreasing. As we have $0 < 1 < 1+5\times 10^{-6} < \pi/2$ we will find our maximum value
	at $cos(1) \approx 0.5403023$. We then multiply this by $\epsilon$ to get the absolute error
	= $\cos(1)\times\epsilon \approx 2.701511529\times10^{-6}$.
	\bigbreak
	Relative error is then $\cos(1)\times\epsilon/\sin(1) \approx 3.210463\times10^{-6}$.

%% Problem 4
\problem{}

	The equation $4x^2 - e^x - e^{-x} = 0$ has four solutions $\pm x_1$ and $\pm x_2$.
	Use Newton's method to approximate the solution to within $10^-5$ with the following values of
	$y_0: \text{(a)} p_0 = -10$, (b) $p_0 = -5$, (c) $p_0 = -3$, (d) $p_0 = -1$, (c) $p_0 = 1$, 
	(g) $p_0 = 3$, (h) $p_0 = 5$, (i) $p_0 = 10$. Comment on the results.

\solution

	\inputminted{octave}{p4.m}

%% Problem 5
\problem{}

	Use Newton's method and Modified Newton's method to find the solution accurate to within $10^-5$ for
	$$
		1 - 4x\cos x + 2x^2 + \cos 2x = 0\text{, for }0 \leq x \leq 1.
	$$
	Comment on the performance of the methods.

\solution

	TODO

%% Problem 6
\problem{}

	Use each of the following methods to find a solution in $[0.1, 1]$ accurate to within $10^-4$ for
	$$
		600x^4 - 550x^3 + 200x^2 - 20x - 1 = 0.
	$$
	\begin{enumerate}
		\item Bisection method
		\item Newton's method
		\item Secant method
		\item M\"{u}ller's method
	\end{enumerate}
	Comment on the performance of these methods.

\solution

	TODO

%% Problem 7
\problem{}

\solution

	An object falling vertically through the air is subjected to viscous resistance as well as the
	force of gravity. Assume that an object with mass $m$ is dropped from a height $s_0$ and that the height of the
	object after $t$ seconds is
	$$
		s(t) = s_0 - \frac{mg}{k}t + \frac{m^2g}{k^2}(1-e^{-kt/m}),
	$$
	where $g=32.17 ft/s^2$ and $k$ represents the coefficient of air resistance in lb-s/ft.
	Suppose $s_0$=300ft, $m$ = 0.25lb, and $k$ = 0.1 lb-s/ft. Find to within 0.01s the time it takes
	this quarter-pounder to hit the ground. Use Fixed-point iteration, Steffensen's method and Newton's method
	to find the solution.

\end{document}
