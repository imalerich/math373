\documentclass[10pt]{jhwhw}
\author{Ian Malerich}
\title{Math S 373: Homework 1}
\usepackage{amssymb, amsfonts, amsmath, mathtools, graphicx, breqn, soul}
\usepackage{minted, subfig, float, scrextend, setspace, amsthm}
\usemintedstyle{friendly}

\begin{document}
\raggedright

%% Problem 1
\problem{} (25 points)

	\begin{enumerate}
		\item Use appropriate Lagrange interpolating polynomials of degrees one, two, and three
			to approximate $f(0.43)$ if
			$$
				f(0) = 1, f(0.25) = 1.64872, f(0.5) = 2.71828, f(0.75) = 4.48169
			$$
		\item Use Neville's method to obtain the approximations for (a). (You may use
			Matlab or other softwares to solve this part. Copies of codes and 
			results must be submitted)
		\item The data in (a) is generated using $f(x) = e^{2x}$. Use the error formula
			to find a bound for the error, and compare the bound to the actual error for the
			cases n=1 and n=2.
		\item Repeat (a) using Newton divided-difference formula. (You may use Matlab
			or other softwares to solve this part. Copies of codes and results must be 
			submitted.)
	\end{enumerate}

\solution

	\part

	For the first degree polynomial we use the points $0.25$ and $0.5$ 
	as we would like to approximate $0.43$ within their bounds.
	\begin{align*}
		L_1(x) &= 1.64872 \frac{(x-0.5)}{0.25-0.5} + 2.71828 \frac{(x-0.25)}{0.5-0.25} \\
		&=\ 4.27824x + 0.57916
	\end{align*}
	\hl{$L_1(0.43) = 2.4188032$} \\

	For the second degree polynomial I could either the first or last point, 
	I found that adding the last point produced a more accurate solution, so I'll use that here.
	\begin{align*}
		L_2(x) &=\ 1\times \frac{(x-0.5)(x-0.75)}{(0.25-0.5)(0.25-0.75)} 
			+ 1.64872\times \frac{(x-0.25)(x-0.75)}{(0.5-0.25)(0.5-0.75)} \\
			&+\ 2.71828\times \frac{(x-0.25)(x-0.5)}{(0.75-0.25)(0.75-0.5)} \\
		&= \ 5.5508^2 + 0.11514x + 1.27301\\
	\end{align*}
	\hl{$L_2(0.43) = 2.34886312$}

	Then for the third degree polynomial, simply use all of the points available.
	\begin{align*}
		L_3(x) &=\ \frac{(x-0.25)(x-0.5)(x-0.75)}	{(-0.25)(-0.5)(-0.75)} + 
			1.64872	\frac{(x-0)(x-0.5)(x-0.75)}		{(0.25)(0.25-0.5)(0.25-0.75)} \\
			&+\ 2.71828	\frac{(x-0)(x-0.25)(x-0.75)}{(0.5)(0.5-0.25)(0.5-0.75)} + 
			0.75	\frac{(x-0)(x-0.25)(x-0.5)}		{(0.75)(0.75-0.25)(0.75-0.5)} \\
		&=\ 2.91211x^3 + 1.18264x^2 + 2.11721x + 1 \\
	\end{align*}
	\hl{$L_3(0.43) = 2.36060356577$}

	\part

	$N_1(0.43) = $\hl{ $2.41880$},
	$N_2(0.43) = $\hl{ $2.34886$},
	$N_3(0.43) = $\hl{ $2.36060$} \\
	Here we can see that the results by Neville's method are equivalent to those
	produced by Lagrange's method, although the Matlab results produced a slightly
	lesser degree of accuracy. The code to produce these results can be found 
	at the bottom of this problem.

	\part

	Firsts, we compute the actual error for each degree of our polynomial. 
	Where $e^{2(0.43)} = 2.36316$ is the actual result. \\
	\begin{align*}
		\text{degree one: } & \lvert\ 2.4188032 - 2.36316\rvert = 0.0556425 \\
		\text{degree two: } & \lvert\ 2.34886312 - 2.36316\rvert = 0.01429688 \\
		\text{degree three: } & \lvert\ 2.36060356577 - 2.36316\rvert = 0.00255643423 \\
	\end{align*}

	The Lagrange polynomial error term is given by 
	$$
	|\frac{f^{(n+1)}(\xi(x))}{(n+1)!}(x-x_0)(x-x_1)\cdots(x-x_n)|
	$$

	To find maximum error, we would like to maximize the term:
	$\lvert f^{(n+1)}(\xi(x))\rvert $.

	\bigbreak
	For degree one we want the maximum on the bounds $(0.25, 0.5)$: \\
	max($|f''(\xi(x))| $) =
	max($|4e^{2(\xi(x))}| $) = $|4e^{2*0.5}|$ = 10.8731. \\
	With just a little more work we have the maximum error as
	$|\frac{10.8731}{2!}(0.43-0.25)(0.43-0.5)| =$ \hl{$0.0685$}. We can see
	that our actual error of $0.0556425$ is just squeezing underneath this maximal value.

	\bigbreak
	Next for degree two, we want the maximum on the bounds $(0.25, 0.75)$: \\
	max($|f^{(3)}(\xi(x))| $) =
	max($|8e^{2(\xi(x))}| $) = $|8e^{2*0.75}|$ = 35.8535. \\
	Then to compute maximum error,
	$|\frac{33.8535}{3!}(0.43-0.25)(0.43-0.5)(0.43-0.75)| =$ \hl{$0.0225877$}.
	This error is obviously improved accuracy over degree one, and again we see that
	our actual error of $0.01429688$ is well within its bounds.

	\bigbreak
	Finally for degree three, we want the maximum on the full bounds $(0, 0.75)$: \\
	max($|f^{(4)}(\xi(x))| $) =
	max($|16e^{2(\xi(x))}| $) = $|16e^{2*0.75}|$ = 71.7070. \\
	Then to compute maximum error,
	$|\frac{71.7070}{4!}(0.43-0)(0.43-0.25)(0.43-0.5)(0.43-0.75)| =$ \hl{$0.00485636$}.
	Again, this behaves as we'd expect, less error than degree two, and our actual
	error of $0.00255643423$ is again well within its bounds.

	\clearpage
	\part

	ndd$_1(0.43)$ = \hl{2.4188} \\
	ndd$_2(0.43)$ = \hl{2.3489} \\
	ndd$_3(0.43)$ = \hl{2.3764} \\

	\bigbreak
	\inputminted{octave}{p2.m}
	\clearpage
	\inputminted{octave}{ndd.m}
	\textcolor[RGB]{240,240,240}{\rule{\textwidth}{0.5pt}}
	\inputminted{octave}{divideddifference.m}
	\textcolor[RGB]{240,240,240}{\rule{\textwidth}{0.5pt}}
	\inputminted{octave}{neville.m}

%% Problem 2
\problem{} (25 points)

	The Bernstein polynomial of degree $n$ for $f\in C[0,1]$ is given by
	$$
		B_n(x) = \Sigma_{k=0}^{n}{n\choose k} f(k/n)x^k (1-x)^{n-k},
	$$
	where ${n\choose k} = n!/(k!(n-k)!)$. These polynomials can be used in a constructive
	proof of Weierstrass Approximation Theorem since $\lim_{n\rightarrow\infty}B_n(x) = f(x)$,
	for each $x\in [0,1]$.
	\begin{enumerate}
		\item Find $B_3(x)$ for the functions: (i) $f(x)=x$ and (ii) $f(x)=1$.
		\item Show that for each $k\leq n$,
			$$
				{n-1\choose k-1} = \frac{k}{n} {n\choose k}
			$$
		\item Use part (b) and the fact,  from (ii) in part (a), that
			$$
				1 = \Sigma_{k=0}^{n} {n\choose k}x^k(1-x)^{n-k}, \forall n \in \mathbb{N},
			$$
			to show that for $f(x) = x^2$,
			$$
				B_n(x) = \frac{n-1}{n}x^2 + \frac{1}{n}x.
			$$
		\item Use part (c) to estimate the value of $n$ necessary for $|B_n(x) - x^2| \leq 10^{-6}$
			to hold for all $x\in [0,1]$.
	\end{enumerate}

\solution

\part

\textbf{f(x) = x} \\
\begin{align*}
	B_3(x) &= 
		{3\choose 0}f(0/3)x^0(1-x)^{3-0}
		+ {3\choose 1}f(1/3)x^1(1-x)^{3-1} \\
		&+ {3\choose 2}f(2/3)x^2(1-x)^{3-2}
		+ {3\choose 3}f(3/3)x^3(1-x)^{3-3} \\
	&= 
		0 \times x^0(1-x)^{3}
		+ 3\times 1/3x^1(1-x)^{2}
		+ 3\times 2/3x^2(1-x)^{1}
		+ x^3(1-x)^{0} \\
	&= 
		x(1-x)^{2}
		+ 2x^2(1-x)
		+ x^3 \\
	&= 
		x-2x^2+x^3
		+ 2x^2 - 2x^3
		+ x^3 \\
	&= 
		x -2x^2 + 2x^2 + 2x^3 - 2x^3 \\
	&= x
\end{align*}

\clearpage
\textbf{f(x) = 1} \\
\begin{align*}
	B_3(x) &= 
		{3\choose 0}f(0/3)x^0(1-x)^{3-0}
		+ {3\choose 1}f(1/3)x^1(1-x)^{3-1} \\
		&+ {3\choose 2}f(2/3)x^2(1-x)^{3-2}
		+ {3\choose 3}f(3/3)x^3(1-x)^{3-3} \\
	&= 
		x^0(1-x)^{3}
		+ 3x^1(1-x)^{2}
		+ 3x^2(1-x)^{1}
		+ x^3(1-x)^{0} \\
	&= 
		-x^3 + 3x^2 - 3x + 1
		+ 3x(1-2x+x^2)
		+ 3x^2(1-x)
		+ x^3 \\
	&= 
		-x^3 + 3x^2 - 3x + 1
		+ 3x - 6x^2 + 3x^3
		+ 3x^2 - 3x^3
		+ x^3 \\
	&= (-x^3 + x^3 + 3x^3 - 3x^3) + (3x^2 - 6x^2 + 3x^2) + (-3x + 3x) + 1 \\
	&= 1 \\
\end{align*}

\part

\begin{proof}

Let $k\leq n$. \\
\begin{align*}
	\binom{n-1}{k-1} &= \frac{(n-1)!}{(k-1)!((n-1)-(k-1))!} = \frac{(n-1)!}{(k-1)!(n-k)!} \\
	&= \frac{k}{k}\frac{(n-1)!}{(k-1)!(n-k)!} = \frac{k(n-1)!}{k!(n-k)!} \\
	&= \frac{n}{n}\frac{k(n-1)!}{k!(n-k)!} = \frac{kn(n-1)!}{nk!(n-k)!} = \frac{kn!}{nk!(n-k)!} \\
	&= \frac{k}{n}\frac{n!}{k!(n-k)!} = \frac{k}{n}\binom{n}{k}
\end{align*}

\end{proof}

\part

\begin{proof}
\begin{align*}
		B_n(x) &= \Sigma_{k=0}^{n}{n\choose k}f(k/n)x^k(1-x)^{n-k} \\
		&= \Sigma_{k=1}^{n}{n\choose k}\frac{k^2}{n^2}x^k(1-x)^{n-k} \\
		&= \frac{x}{n}\Sigma_{k=1}^{n}{n\choose k}\frac{k^2}{n}x^{k-1}(1-x)^{n-k} \\
		&= \frac{x}{n}\Sigma_{k=1}^{n}{n-1\choose k-1}kx^{k-1}(1-x)^{n-k} \\
		&= \frac{x}{n}\Sigma_{k=0}^{n-1}{n-1\choose k}(k+1)x^{k}(1-x)^{n-k} \\
		&= \frac{x}{n}\Sigma_{k=0}^{n-1}({n-1\choose k}kx^{k}(1-x)^{n-k}
			+ {n-1\choose k}x^k(1-x)^{(n-1)-k}) \\
		&= \frac{x}{n}(\Sigma_{k=0}^{n-1}{n-1\choose k}kx^{k}(1-x)^{n-k}
			+ \Sigma_{k=0}^{n-1}{n-1\choose k}x^k(1-x)^{(n-1)-k}) \\
		&= \frac{x}{n}(\Sigma_{k=0}^{n-1}({n-1\choose k}kx^{k}(1-x)^{n-k})
			+ 1) \\
		&= \frac{x}{n}(\Sigma_{k=0}^{n-1}(\frac{(n-1)!}{k!((n-1)-k)!}kx^{k}(1-x)^{n-k})
			+ 1) \\
		&= \frac{x}{n}(\Sigma_{k=0}^{n-1}(\frac{(n-1)!}{(k-1)!((n-2)-(k-1))!}x^{k}(1-x)^{n-k})
			+ 1) \\
		&= \frac{x}{n}(x(n-1)\Sigma_{k=0}^{n-1}(\frac{(n-2)!}{(k-1)!((n-2)-(k-1))!}x^{k-1}(1-x)^{n-k})
			+ 1) \\
		&= \frac{x}{n}(x(n-1)(1) + 1) \\
		&= \frac{x}{n}(x(n-1) + 1) \\
		&= \frac{x^2(n-1) + x}{n} \\
		&= \frac{(n-1)}{n}x^2 + \frac{1}{n}x \\
\end{align*}
\end{proof}

\clearpage
\part

\begin{align*}
	|\frac{n-1}{n}x^2 + \frac{1}{n}x - x^2| &\leq 10^{-6} \Rightarrow \\
	|(\frac{n-1}{n} - \frac{n}{n})x^2 + \frac{x}{n}| &\leq 10^{-6} \Rightarrow \\
	|\frac{-x^2}{n} + \frac{x}{n}| &\leq 10^{-6} \Rightarrow \\
	|\frac{x-x^2}{n}| &\leq 10^{-6} \Rightarrow \\
	\frac{1}{n}|x-x^2| &\leq 10^{-6}
\end{align*}

We want this to hold $\forall x\in [0,1]$ thus we need the maximum value of $|x-x^2|$ on
those bounds. Thus is found at $x=0.5$ with $|0.5-0.5^2| = 0.25$. Thus we want
$\frac{1}{4n} \leq 10^{-6} \Rightarrow 4n \geq 10^6 \Rightarrow n \geq \frac{10^6}{4} 
\Rightarrow n \geq$ \hl{250000}.

%% Problem 3
\problem{} (10 points)

	\begin{enumerate}
		\item Show that the cubic polynomials
			$$
				P(x) = 3 - 2(x+1) + 0(x+1)(x) + (x+1)(x)(x-1)
			$$
			and
			$$
				Q(x) = -1 + 4(x+2) - 3(x+2)(x+1) + (x+2)(x+1)(x)
			$$
			both interpolate the data
			$f(-2) = -1, f(-1) = 3, f(0) = 1, f(1) = -1, f(2) = 3$
		\item Why does part (a) not violate the uniqueness property of interpolating 
			polynomials?
	\end{enumerate}

\solution

\part
	Using the MATLAB code provided below, we can see that both $P(x)$ and $Q(x)$ produce
	the expected results. That is $P(-2) = Q(-2) = f(-2) = -1, 
	P(-1) = Q(-1) = f(-1) = 3, P(0) = Q(0) = f(0) = 1, P(1) = Q(1) = f(1) = -1,$ and 
	$P(2) = Q(2) = f(2) = 3$. \\
	Thus the given cubic polynomials both clearly interpolate the provided points of $f$.

	\inputminted{octave}{p3.m}

\part
	Because 5 interpolating points are provided for $f$, where as $P$ and $Q$ only
	have degree $3$. The uniqueness property of interpolating polynomials would only
	holds for polynomials of degree $4$.

%% Problem 4
\problem{} (20 points)

	(You may use Matlab or other softwares to solve this problem. Copies of codes and results
	must be submitted.)
	Let $f(x) = 3xe^x - e^{2x}$.
	\begin{enumerate}
		\item Approximate $f(1.03)$ by the Hermite interpolating polynomial of dgree at most
			three using $x_0 = 1$ and $x_1 = 1.05$. Compare the actual error to the error bound.
		\item Repeat (a) with the Hermite interpolating polynomial of degree at most five,
			using $x_0=1, x_1=1.05$ and $x_2 = 1.07$.
	\end{enumerate}

\solution

\part

The actual result is $f(1.03) = 0.80932$ with the Hermite estimate being 
	$H(1.03) = 0.80932$. At the precision MATLAB outputs these seem nearly identical,
	however the code which computes the error gives us a small error between
	the two results with ERROR = $1.2373\times10^{-6}$.

The error term of the Hermite polynomial is given by
$
	|\frac{f^{(2n+2)\xi(x)}}{(2n+2)!}\Pi_{k=0}^{n}(x-x_i)^2|
$
We only have $n=1$, so we will need the 4th derivative, this is given by 
$f^{(4)}(x) = e^x(3(x+4) - 16e^x)$. Maximizing error requires maximizing the magnitude
on our bounds $[1, 1.05]$. This is found at $x=1.05$ where $|f^(4)(1.05)| = 87.3653$.
Thus, our maximum error will be
$$
	|\frac{87.3653}{4!}(1.03-1)^2(1.03-1.05)^2| = 1.3104795\times 10^{-6}
$$
This is very close to the actual error we discovered, about $1.2\times 10^{-6}$ actual
versus the $1.3\times 10^{-6}$ maximum theoretical.

\inputminted{octave}{p4a.m}

\part

Again the actual result of $f(1.03) = 0.80932$ is equal to the Hermite estimate
	of $H(1.03) = 0.80932$ due to the limitations of MATLAB output. But when 
	we produce an error, we see that it has improved over the result in part (a),
	with an error of only $3.6101\times10^{-10}$.

The error term of the Hermite polynomial is given by
$
	|\frac{f^{(2n+2)\xi(x)}}{(2n+2)!}\Pi_{k=0}^{n}(x-x_i)^2|
$
This time we have $n=2$, so we will need the 6th derivative, this is given by 
$f^{(6)}(x) = 6e^x(x+6)-64e^{2x}$. Maximizing error requires maximizing the magnitude
on our bounds $[1, 1.07]$. This is found at $x=1.07$ where $|f^(6)(1.07)| = 420.294$.
Thus, our maximum error will be
$$
	|\frac{420.294}{6!}(1.03-1)^2(1.03-1.05)^2(1.03-1.07)^2| = 1.73699\times 10^{-4}
$$
Although the actual error did decrease from part (a) to part (b), widening the bounds
did actually increase the maximum theoretical error.

\inputminted{octave}{p4b.m}
\clearpage
\textcolor[RGB]{240,240,240}{\rule{\textwidth}{0.5pt}}
\inputminted{octave}{Hermiteinterpolation.m}

%% Problem 5
\problem{} (20 points)

	\begin{enumerate}
		\item Determine the free cubic spline $S$ that interpolates the data $f(0) = 0,$
			$f(1) =1$ and $f(2)=2$.
		\item Determine the clamped cubic spline $s$ that interpolates the data
			$f(0) = 0, f(1) = 1, f(2) = 2$ and satisfies $s'(0) = s'(2) = 1$.
	\end{enumerate}

\solution

\part

Using the MATLAB code included below, we can produce the coefficients for
each of the cubic polynomials which define the natural cubic spline $S$. 
$S$ is then given by
$$
S(x) = 
\begin{cases}
	0 + 1x + 0x^2 + 0x^3 : 0 \leq x \leq 1\\
	1 + 1x + 0x^2 + 0x^3 : 1 < x \leq 2\\
\end{cases}
$$

\part

Running similar code, but for the clamped cubic spline with derivatives of 1
produces the same piece-wise cubic polynomials.
$$
S(x) = 
\begin{cases}
	0 + 1x + 0x^2 + 0x^3 : 0 \leq x \leq 1\\
	1 + 1x + 0x^2 + 0x^3 : 1 < x \leq 2\\
\end{cases}
$$

\inputminted{octave}{p5.m}

\end{document}
