\documentclass[10pt]{jhwhw}
\author{Ian Malerich}
\title{Math S 373: Homework 1}
\usepackage{amssymb, amsfonts, amsmath, mathtools, graphicx, breqn, soul}
\usepackage{minted, subfig, float, scrextend, setspace, amsthm}
\usemintedstyle{friendly}

\begin{document}
\raggedright

%% Problem 1
\problem{} (25 points)

	\begin{enumerate}
		\item Use appropriate Lagrange interpolating polynomials of degrees one, two, and three
			to approximate $f(0.43)$ if
			$$
				f(0) = 1, f(0.25) = 1.64872, f(0.5) = 2.71828, f(0.75) = 4.48169
			$$
		\item Use Neville's method to obtain the approximations for (a). (You may use
			Matlab or other softwares to solve this part. Copies of codes and 
			results must be submitted)
		\item The data in (a) is generated using $f(x) = e^{2x}$. Use the error formula
			to find a bound for the error, and compare the bound to the actual error for the
			cases n=1 and n=2.
		\item Repeat (a) using Newton divided-difference formula. (You may use Matlab
			or other softwares to solve this part. Copies of codes and results must be 
			submitted.)
	\end{enumerate}

\solution

	\part

	For the first degree polynomial we use the points $0.25$ and $0.5$ 
	as we would like to approximate $0.43$ within their bounds.
	\begin{align*}
		L_1(x) &= 1.64872 \frac{(x-0.5)}{0.25-0.5} + 2.71828 \frac{(x-0.25)}{0.5-0.25} \\
		&=\ 4.27824x + 0.57916
	\end{align*}
	\hl{$L_1(0.43) = 2.4188032$} \\

	For the second degree polynomial I could either the first or last point, 
	I found that adding the last point produced a more accurate solution, so I'll use that here.
	\begin{align*}
		L_2(x) &=\ 1\times \frac{(x-0.5)(x-0.75)}{(0.25-0.5)(0.25-0.75)} 
			+ 1.64872\times \frac{(x-0.25)(x-0.75)}{(0.5-0.25)(0.5-0.75)} \\
			&+\ 2.71828\times \frac{(x-0.25)(x-0.5)}{(0.75-0.25)(0.75-0.5)} \\
		&= \ 5.5508^2 + 0.11514x + 1.27301\\
	\end{align*}
	\hl{$L_2(0.43) = 2.34886312$}

	Then for the third degree polynomial, simply use all of the points available.
	\begin{align*}
		L_3(x) &=\ \frac{(x-0.25)(x-0.5)(x-0.75)}	{(-0.25)(-0.5)(-0.75)} + 
			1.64872	\frac{(x-0)(x-0.5)(x-0.75)}		{(0.25)(0.25-0.5)(0.25-0.75)} \\
			&+\ 2.71828	\frac{(x-0)(x-0.25)(x-0.75)}{(0.5)(0.5-0.25)(0.5-0.75)} + 
			0.75	\frac{(x-0)(x-0.25)(x-0.5)}		{(0.75)(0.75-0.25)(0.75-0.5)} \\
		&=\ 2.91211x^3 + 1.18264x^2 + 2.11721x + 1 \\
	\end{align*}
	\hl{$L_3(0.43) = 2.36060356577$}

	\part

	$N_1(0.43) = $\hl{ $2.41880$},
	$N_2(0.43) = $\hl{ $2.34886$},
	$N_3(0.43) = $\hl{ $2.36060$} \\
	Here we can see that the results by Neville's method are equivalent to those
	produced by Lagrange's method, although the Matlab results produced a slightly
	lesser degree of accuracy. The code to produce these results can be found 
	at the bottom of this problem.

	\part

	Firsts, we compute the actual error for each degree of our polynomial. 
	Where $e^{2(0.43)} = 2.36316$ is the actual result. \\
	\begin{align*}
		\text{degree one: } & \lvert\ 2.4188032 - 2.36316\rvert = 0.0556425 \\
		\text{degree two: } & \lvert\ 2.34886312 - 2.36316\rvert = 0.01429688 \\
		\text{degree three: } & \lvert\ 2.36060356577 - 2.36316\rvert = 0.00255643423 \\
	\end{align*}

	The Lagrange polynomial error term is given by 
	$$
	|\frac{f^{(n+1)}(\xi(x))}{(n+1)!}(x-x_0)(x-x_1)\cdots(x-x_n)|
	$$

	To find maximum error, we would like to maximize the term:
	$\lvert f^{(n+1)}(\xi(x))\rvert $.

	\bigbreak
	For degree one we want the maximum on the bounds $(0.25, 0.5)$: \\
	max($|f''(\xi(x))| $) =
	max($|4e^{2(\xi(x))}| $) = $|4e^{2*0.5}|$ = 10.8731. \\
	With just a little more work we have the maximum error as
	$|\frac{10.8731}{2!}(0.43-0.25)(0.43-0.5)| =$ \hl{$0.0685$}. We can see
	that our actual error of $0.0556425$ is just squeezing underneath this maximal value.

	\bigbreak
	Next for degree two, we want the maximum on the bounds $(0.25, 0.75)$: \\
	max($|f^{(3)}(\xi(x))| $) =
	max($|8e^{2(\xi(x))}| $) = $|8e^{2*0.75}|$ = 35.8535. \\
	Then to compute maximum error,
	$|\frac{33.8535}{3!}(0.43-0.25)(0.43-0.5)(0.43-0.75)| =$ \hl{$0.0225877$}.
	This error is obviously improved accuracy over degree one, and again we see that
	our actual error of $0.01429688$ is well within its bounds.

	\bigbreak
	Finally for degree three, we want the maximum on the full bounds $(0, 0.75)$: \\
	max($|f^{(4)}(\xi(x))| $) =
	max($|16e^{2(\xi(x))}| $) = $|16e^{2*0.75}|$ = 71.7070. \\
	Then to compute maximum error,
	$|\frac{71.7070}{4!}(0.43-0)(0.43-0.25)(0.43-0.5)(0.43-0.75)| =$ \hl{$0.00485636$}.
	Again, this behaves as we'd expect, less error than degree two, and our actual
	error of $0.00255643423$ is again well within its bounds.

	\clearpage
	\part

	ndd$_1(0.43)$ = \hl{2.4188} \\
	ndd$_2(0.43)$ = \hl{2.3489} \\
	ndd$_3(0.43)$ = \hl{2.3764} \\

	\bigbreak
	\inputminted{octave}{p2.m}
	\clearpage
	\inputminted{octave}{ndd.m}
	\textcolor[RGB]{240,240,240}{\rule{\textwidth}{0.5pt}}
	\inputminted{octave}{divideddifference.m}
	\textcolor[RGB]{240,240,240}{\rule{\textwidth}{0.5pt}}
	\inputminted{octave}{neville.m}

%% Problem 2
\problem{} (25 points)

	The Bernstein polynomial of degree $n$ for $f\in C[0,1]$ is given by
	$$
		B_n(x) = \Sigma_{k=0}^{n}{n\choose k} f(k/n)x^k (1-x)^{n-k},
	$$
	where ${n\choose k} = n!/(k!(n-k)!)$. These polynomials can be used in a constructive
	proof of Weierstrass Approximation Theorem since $\lim_{n\rightarrow\infty}B_n(x) = f(x)$,
	for each $x\in [0,1]$.
	\begin{enumerate}
		\item Find $B_3(x)$ for the functions: (i) $f(x)=x$ and (ii) $f(x)=1$.
		\item Show that for each $k\leq n$,
			$$
				{n-1\choose k-1} = \frac{k}{n} {n\choose k}
			$$
		\item Use part (b) and the fact,  from (ii) in part (a), that
			$$
				1 = \Sigma_{k=0}^{n} {n\choose k}x^k(1-x)^{n-k), \text{ for each } n,
			$$
			to show that for $f(x) = x^2$,
			$$
				B_n(x) = \frac{n-1}{n}x^2 + \frac{1}{n}x.
			$$
		\item Use part (c) to estimate the value of $n$ necessary for $|B_n(x) - x^2| \leq 10^{-6}$
			to hold for all $x\in [0,1]$.
	\end{enumerate}

\solution

	TODO

%% Problem 3
\problem{} (10 points)

	\begin{enumerate}
		\item Show that the cubic polynomials
			$$
				P(x) = 3 - 2(x+1) + 0(x+1)(x) + (x+1)(x)(x-1)
			$$
			and
			$$
				Q(x) = -1 + 4(x+2) - 3(x+2)(x+1) + (x+2)(x+1)(x)
			$$
			both interpolate the data
			$f(-2) = -1, f(-1) = 3, f(0) = 1, f(1) = -1, f(2) = 3$
		\item Why does part (a) not violate the uniqueness property of interpolating 
			polynomials?
	\end{enumerate}

\solution

	TODO

%% Problem 4
\problem{} (20 points)

	(You may use Matlab or other softwares to solve this problem. Copies of codes and results
	must be submitted.)
	Let $f(x) = 3xe^x - e^{2x}$.
	\begin{enumerate}
		\item Approximate $f(1.03)$ by the Hermite interpolating polynomial of dgree at most
			three using $x_0 = 1$ and $x_1 = 1.05$. Compare the actual error to the error bound.
		\item Repeat (a) with the Hermite interpolating polynomial of degree at most five,
			using $x_0=1, x_1=1.05$ and $x_2 = 1.07$.
	\end{enumerate}

\solution

	TODO

%% Problem 5
\problem{} (20 points)

	\begin{enumerate}
		\item Determine the free cubic spline $S$ that interpolates the data $f(0) = 0,$
			$f(1) =1$ and $f(2)=2$.
		\item Determine the clamped cubic spline $s$ that interpolates the data
			$f(0) = 0, f(1) = 1, f(2) = 2$ and satisfies $s'(0) = s'(2) = 1$.
	\end{enumerate}

\solution

	TODO

\end{document}
